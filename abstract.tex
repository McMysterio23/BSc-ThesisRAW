\chapter*{Abstract}

%%%CORREZIONI DI ANDREA APPLICATE TUTTE !!!
Brightest cluster galaxies (BCGs) are the most massive and luminous galaxies
located near the center of relaxed, virialized, and undisturbed galaxy clusters in the local Universe
(\cite{1976ApJ...205..688S, 2010MNRAS.404.1231V}). According to several observational studies
(\cite{2017MNRAS.467.4101G, 2020MNRAS.498.2719T}), these objects experience a special formation
process differing from general galaxy evolution.

Current theoretical models (e.g., \cite{2007MNRAS.375....2D, 2019ApJ...881..150C}) predict that dry
mergers are the dominant mechanisms responsible for their mass assembly at $z < 1$. These
objects are often observed to host a supermassive black hole (SMBH) in their center (\cite{2006ApJ...652..216R}).
The process of matter accretion into these SMBHs may release a large amount of
energy, resulting in Active Galactic Nuclei (AGN). There are two primary modes in which SMBH
accretion can occur: the so-called ’quasar mode’ and the ’radio mode’. The quasar mode involves
a high accretion rate of the SMBH via an optically- thick and geometrically-thin disk, with most of
the energy being released in the form of radiation. In a radio mode scenario, the SMBH accretion
of gas occurs at a low rate with an optically-thin and geometrically-thick disk configuration,
releasing energy in the form of relativistic particles, i.e. radio jets. The latter is typically observed in
BCGs \cite{2012MNRAS.422.2213S}.

The evolutionary processes of BCGs are still not fully understood, and there are no specific
studies comparing the frequency of different types of AGN in BCGs with respect to other types of
galaxies (e.g., \cite{2019CoBAO..66..153F}). The main scientific question guiding my thesis project is to
investigate whether the different evolution of BCGs, coupled with their ”special” environment,
promotes the accretion of SMBHs in their centers compared to other types of galaxies in the local
universe, at $z < 0.1$. To address this question, I analyzed a sample of BCGs within the redshift
range of z = 0.02-0.1. This sample was derived from the combination of the Sloan Digital Sky
Survey Data Release 7 (SDSS DR7 \cite{2009ApJS..182..543A}) and the
C4 BCGs catalogue produced by \cite{2007MNRAS.379..867V, 2009yCat..73790867V}. I utilized the flux measurements of optical
emission lines i.e. $H\alpha$, [OIII], $H\beta$, [NII], and [SII] doublets estimated by the Max Planck Institute for Astrophysics and Johns Hopkins University ( MPA-JHU ) teams using the methods outlined in \cite{1994ApJ...422..158O}.
This allowed me to conduct a selection of
optical AGN through the [NII]- and [SII]- BPT diagnostic diagrams  \cite{1981PASP...93....5B}.
Additionally, I conducted a cross-matching of the aforementioned catalogs with a dataset
obtained from \cite{2005MNRAS.362....9B}, where the spectroscopic sample of the SDSS DR2 was cross-correlated with catalogs of galaxies observed from the National Radio Astronomy Observatory
(NRAO) Very Large Array (VLA) Sky Survey (NVSS; \cite{1998AJ....115.1693C}) and the Faint Images of
the Radio Sky at Twenty centimeters (FIRST) survey \cite{1995ApJ...450..559B}.
Using this new catalog of BCGs probed with these radio surveys, I was able to select the BCGs
that exhibit radio loudness.
Following this classification, I finally estimated the fraction of BCGs
classified as Optical and Radio Loud AGN. Subsequently, I derived these fractions for a sample of
non-BCG selected galaxies using the same procedure employed to obtain the BCG catalog.

These analyses reveal that BCGs exhibit a higher fraction of Optical AGN, in a range of 58$\%$ to 74$\%$, compared to the
non-BCG sample, which shows a lower percentage $\sim13\%$, consistent with the results found by \cite{2012A&A...546A..17V}.
Simultaneously, the analysis of Radio Loud emissions indicates that BCGs are more
inclined to host Radio Loud Activity, with a fraction of  $\sim12\%$.
This fraction is found to be 20 times higher than the fraction observed in 
the non-BCG sample of selected galaxies, which is  $\sim0.6\%$.

In conclusion, these results demonstrate that BCGs are more likely to host optical AGN activity
and radio-loud emission compared to other types of galaxies.
This suggests that their privileged position facilitates frequent accretion of SMBHs in both
accretion modes. Previous studies (e.g.  \cite{2016MNRAS.460.3669Y, 2014MNRAS.440..762O, 2007MNRAS.379..867V}) have already shown a prevalence of radio AGN among the BCG
population in the local Universe. On the other hand, few studies (e.g., \cite{2019CoBAO..66..153F, 2007MNRAS.379..867V}) have attempted to
compare the distribution of BCGs in BPT diagrams to that of normal galaxies. The fact that the
fraction of AGN is greater for special galaxies is important to understand the
nature of these objects. Future studies will aim to test the results obtained with this sample and to
understand if this higher fraction is specifically driven by differences in properties between BCGs
and non-BCGs, such as mass, star formation rate, metallicity, and kinematics.

