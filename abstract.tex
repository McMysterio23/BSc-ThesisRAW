\chapter*{Abstract}
Brightest cluster galaxies (BCGs) are the most massive and luminous galaxies located near the center of relaxed, virialized, and undisturbed galaxy clusters in the local Universe (\cite{1976ApJ...205..688S}; \cite{2010MNRAS.404.1231V}). 

According with several observational studies (\cite{2020MNRAS.498.2719T}) these objects experience special formation differing from galaxy evolution in general. Current theoretical models (e.g., \cite{2007MNRAS.375....2D}, \cite{2019ApJ...881..150C}) predict that the dry mergers are the dominant mechanisms responsible for their mass assembly.

Often, these objects are observed to host a super-massive black hole (SMBH) in their center (\cite{2006ApJ...652..216R}).  
The process of matter accretion into these SMBHs may release a large amount of energy, resulting in Active Galactic Nuclei (AGN).
There are two primary modes in which SMBH accretion can occur: the so-called 'radio mode' and the 'quasar mode'
The quasar mode consists in a high accretion rate of the SMBH via an optically-thick and geometrically-thin disk, and most of the energy is released in form of radiation. In a radio mode scenario, the SMBH accretion of hotter gas happens with a low rate in a optically-thin and geometrically-thick disk configuration, releasing energy in form of relativistic particles. The latter is typically observed in BCG. 

%The fraction of AGN activity in BCGs has been explored in previous papers (\cite{2016MNRAS.460.3669Y}, \cite{2014MNRAS.440..762O}), showing that the radio-loud fractions increase together with the source magnitude, while on the optical side, BCGs are more likely to host AGN activity rather than Star Forming Character.

Therefore, in this thesis i present an analysis of Optical and Radio Loud AGN fractions on 2 galaxy samples derived from the combination of the galaxy sample created by the MPA-JHU team in the context of  "The Sloan Digital Sky Survey" ( SDSS ),  the C4 BCGs catalogue by \cite{2005AJ....130..968M}, and the Radio Emitters catalogue developed by \cite{2005MNRAS.362....9B}.

As the Optical activity is classified through the BPT diagram diagnostic, the sample without the BCGs has been obtained by overlapping space regions covered by both the SDSS survey and the Radio Emitters one, to achieve a proper confrontation between the results in the different samples.

These analyses reveal that based on a optical point of view BCGs show a greater AGN activity $\sim 60 \%$ when compared to the $\sim 16\%$ obtained on the other sample similar to the ones obtained by \cite{2012A&A...546A..17V}.
While At the same time Radio Loud inspections show that the fractions of AGN objects is similar between the BCG sample ($\sim 62 \%$) and the other ( $\sim 68\%$)

In conclusion, these results strongly support the idea that BCGs are more likely to exhibit radio loudness, as observed in previous studies such \cite{2016MNRAS.460.3669Y, 2014MNRAS.440..762O}. Therefore, any future investigations in this context should prioritize a radio analysis over an optical one.



