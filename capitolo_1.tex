\chapter{Introduction}

\section{The Active Galactic Nuclei Feedback}

\section{The Brightest Cluster Galaxies }
One of the main branches of cosmic structure formation is covered by the Hierarchical model in which galaxies grow in both stellar and mass magnitude, by accreting the surrounding matter.

One of the most extreme examples in this context involves the study of Brightest Cluster Galaxies (BCGs), a unique class of galaxies typically elliptical, often situated at the center and typically standing out as the most luminous and massive objects within the entire cluster.( e.g. \cite{2015MNRAS.448....2W} ).

Considering the environment, also their evolution is proven via observational studies e.g. \cite{2020MNRAS.498.2719T}, to be slightly different to normal galactic ones, resulting in the current main model assumes that BCGs mass assembly is dominated by dry mergers  \cite{2007MNRAS.375....2D, 2019ApJ...881..150C}.


Thanks to their exclusive environment in which they live, consisting in a BCGs often host a Super Massive Black Hole ( SMBH ) (\cite{2006ApJ...652..216R}) resulting in Active Galactic Nuclei ( AGN ) activity such as Radio Jets that feedback into the Intra-Cluster Medium (ICM)\cite{2007ARA&A..45..117M}.

As seen in previous studies ( e.g. \cite{2007MNRAS.379..867V} ) BCGs more likely host Radio Loud Emission than in other type of galaxies, as a result of the influence of the environment other than its nuclei.


% Aggiungi parte sulle merging activity
% aggiungi parte su cosa si vede a bassi z e quali sono le teorie in questo senso che tengono banco
% Aggiungi dei riferimenti alle domande che restano aperte nel contesto.



%Thanks to the unique environment in which BCGs are located, made by over-dense 

\section{The Aim of this thesis}

