\chapter{Methods}
The study of cosmic objects, such as galaxies, relies on the analysis of the electromagnetic spectrum emanating from these distant sources. Within a spectrum, various pieces of information can be extracted, with the primary focus being on the intensity of light across a range of energies or frequencies. A crucial aspect of a spectrum involves determining the intensity at specific wavelengths.

This thesis will concentrate on utilizing specific information derived from spectra (e.g., flux, flux error) associated with both permitted and forbidden emission lines. In a galactic context, particularly within a galaxy cluster, the continuum originates from the diffuse light emitted by stars. On the other hand, emission lines, which are prominently observed in such environments, are typically generated by elements like Hydrogen, Helium, Oxygen, etc.

There are various methods to investigate the electromagnetic spectrum, with the two primary branches being Spectroscopy and Photometry.

Astrophysical spectroscopy is a fundamental tool used to analyze the electromagnetic radiation emitted or absorbed by celestial objects. By examining the spectral lines and features, it is possible to mine important informations such as chemical composition, temperature, density etc.

On the other hand, photometry measures the overall brightness of celestial objects across different wavelength bands, providing information about the spectral distribution of their luminous emission. 

In this Thesis Work i'll be focusing on a Spectroscopic study involving the fluxes of the Forbidden Emission lines of the spectrum as presented in the following sections.

\section{Data Description}
This thesis presents results obtained through the cross-matching of three distinct celestial catalogues:
\begin{itemize}
	\item\textbf{SDSS DR7 :} Our Main Catalogue of galaxies
	\item \textbf{C4-BCG :} The BCG Catalog 
	\item \textbf{Radio Emitters :}  The Sample used for RadioLoud identification
\end{itemize}

\subsection{SDSS DR7}

 
\begin{comment}
1. **Introduzione al SDSS:**
   - Anno di avvio e periodo operativo.
   - Scopo principale del progetto.

2. **Strumenti e Telescopio:**
   - Tipo di telescopio utilizzato.
   - Strumenti di rilevamento e le loro caratteristiche principali.

3. **Area Coperta:**
   - Estensione dell'area del cielo coperta dal SDSS.
   - Dettagli sulla suddivisione dell'area (ad esempio, Northern e Southern Galactic Cap).

6. **Catalogo Galattico:**
   - Informazioni sulla componente galattica del catalogo.
   - Parametri inclusi per le galassie.

8. **Dati Spettrali:**
   - Presenza di dati spettrali nel SDSS.
   - Risoluzione spettrale e gamma di lunghezze d'onda coperte.

9. **Survey Data Releases:**
   - Divisione dei dati in "Data Releases" e principali miglioramenti/funzionalità introdotte in ciascuno.




1. **Introduzione al Catalogo C4:**
   - Breve presentazione del catalogo C4.
   - Contesto in cui è stato sviluppato.

2. **Origine del Catalogo:**
   - Motivazioni e scopi per cui è stato creato.
   - Data di pubblicazione e contesto scientifico.

3. **Selezione delle BCG:**
   - Criteri utilizzati per identificare le Brightest Cluster Galaxies (BCG).
   - Parametri considerati nella selezione.
   - Criteri per l'eliminazione dei doppioni (Opzionale)

4. **Parametri Inclusi per le BCG:**
   - Elenco dei parametri forniti per ciascuna BCG.
   - Possibili caratteristiche fotometriche e spettrali.

5. **Copertura del Cielo:**
   - Estensione dell'area del cielo coperta dal catalogo.
   - Suddivisione geografica, se pertinente.

6. **Risultati Chiave:**
   - Eventuali risultati scientifici ottenuti utilizzando il catalogo.
   - Contributi specifici alle conoscenze astronomiche.


8. **Accesso ai Dati:**
   - Aggiungere il riferimento alla bibliografia !!

\end{comment}
\newpage
\section{Data Analysis}
How we analyzed the data, where do we defined the fractions calculated ?
\subsection{Optical analysis}
Describe each population of the BPT diagram and its peculiarities, before starting the description of how we collected and processed the data to finally create BPT diagrams .
Finally there's the need to explain how we interpreted the diagrams to finally calculate the fractions ( in this case you need to explain which algorithm has been chosen and why ) 


\subsection{Radio Analysis}
I certainly need to focus on how i have chosen the range of identification, to recognize which element of sdss was effectively recognized as a Radioloud.

Following to this it is necessary to explain how we selected the elements in which define the fractions, by selecting elements of SDSS nearby the location in which best et al finds mostly its radio-emitting elements!