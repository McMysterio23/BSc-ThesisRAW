\chapter{Results}
At the conclusion of the analysis, a comparison between the results obtained in the two samples was conducted, and the outcomes are presented in the following tables.

In Table \ref{tab:Optical}, detailing the results of the optical analysis, it is observed that BCGs are more likely to host an AGN compared to ordinary galaxies. These findings align with those presented in the work of Vitale et al. \cite{2012A&A...546A..17V}, where the analysis was based on a statistically larger sample than the one employed in this study.

\vspace{1cm}
\begin{table}[htb]
  \centering
  \begin{tabular}{||c|ccc||}
    \hline\hline
    \multicolumn{1}{||c}{Diagram} & Spectral Type & SDSS DR7 subsample & Crossmatch C4-BCG \\
    \hline
    [NII]  & AGNs &  $52896 \pm 84$ ($21.1 \pm 0.03$)$\%$ &$161 \pm 5$ ($50.4 \pm 1.6$)$\%$ \\
           & Composites & $55875 \pm 110$ ($22.3 \pm 0.04$)$\%$ & $110 \pm 5$ ($34.5 \pm 1.8$)$\%$ \\
           & SFGs & $141753 \pm 80$ ($56.6 \pm 0.03$)$\%$ &$ 49 \pm 4$ ($15.4 \pm 1.4$)$\%$ \\ 
    \hline
    [SII]  & Seyferts & $13082 \pm 64$ ($6.3 \pm 0.03$)$\%$  &$8 \pm 2$ ($6.5 \pm 1.6$)$\%$\\
           & LINERs & $28396 \pm 80$ ($13.6 \pm 0.04$)$\%$ & $74 \pm 3$ ($53.3 \pm 2.4$)$\%$ \\
           & SFGs &$167826\pm 76 $ ($80.2 \pm 0.04$)$\%$ & $55 \pm 3$ ($40.2 \pm 2.2$)$\%$\\ 
    \hline\hline
  \end{tabular}
  \caption{Results of the optical analysis}
  \label{tab:Optical}
\end{table}



Simultaneously, the analysis of Radio Loud emissions indicates that BCGs are more likely to host Radio Loud Activity, with a fraction of $12\%$. This value is 20 times higher than the fraction found for the non-BCG subsample of galaxies within the selected regions, as discussed earlier, corresponding to $0.6\%$.
