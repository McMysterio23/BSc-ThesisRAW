\documentclass[12pt,twoside,a4paper,fleqn,openright]{book}


% 11pt         e' la dimensione del carattere
% twoside      e' per avere fronte/retro
% a4paper      si spiega da solo
% fleqn        e' per avere le equazioni allineate a sx
% openright    e' per iniziare i capitoli sempre a destra

\usepackage[english]{babel}  % per la lingua: cambia il nome dei capitoli in italiano
\usepackage[dvips]{graphicx}         % per inserire le immagini
\usepackage{verbatim}                % per usare \begin{comment}
\usepackage{syntonly}                % per usare il comando syntaxonly
%\usepackage{subfigure}               % per strutturare le immagni
\usepackage{fancyhdr}           % per la righetta fottuta & co.
%\usepackage{rotating}                % chissa' a che serve?

%\documentclass[12pt,a4paper,]{article}
\usepackage[utf8]{inputenc}
\usepackage{mathtools}
\usepackage{listings}
\usepackage{verbatim}
\usepackage{graphicx}
\usepackage{setspace}
\usepackage{xcolor}
\usepackage{wrapfig}
%\usepackage[demo]{graphicx}
%\usepackage{caption}
\usepackage{subcaption}
\usepackage{caption}
\usepackage[utf8]{inputenc}
\usepackage{mathtools}
\usepackage{listings}
\usepackage{verbatim}
\usepackage{graphicx}
\graphicspath{ {./images/} }
\usepackage{float}
\usepackage{mwe}
\usepackage{tabularx}


%syntaxonly       % per eseguire solo un controllo sintattico, senza compilare
\linespread{1.2}  % Per avere una maggiore interlinea. Se si dovesse cambiare, attenzione ai 
                  % comandi dati per l'interlinea delle caption. 

%\author{Andrea Maccarinelli}
%\title{TESI DI LAUREA TRIENNALE}


%
\documentclass[a4]{article}
\usepackage[utf8]{inputenc}
\usepackage{mathtools}
\usepackage{listings}
\usepackage{verbatim}
\usepackage{graphicx}
\usepackage{setspace}
\usepackage{xcolor}
\usepackage{wrapfig}
%\usepackage[demo]{graphicx}
%\usepackage{caption}
\usepackage{subcaption}
\usepackage{caption}
\usepackage[utf8]{inputenc}
\usepackage{mathtools}
\usepackage{listings}
\usepackage{verbatim}
\usepackage{graphicx}
\graphicspath{ {./images/} }
\usepackage{float}
\usepackage{mwe}
\usepackage{tabularx}
\setlength{\headheight}{14.49998pt}
\addtolength{\topmargin}{-2.49998pt}

\newcommand{\HRule}{\rule{\linewidth}{0.5mm}} % Defines a new command for the horizontal lines, change thickness here

%\renewcommand{\baselinestretch}{1.05} 

\thispagestyle{empty}
\begin{document}



\center % Center everything on the page
 
%----------------------------------------------------------------------------------------
%  HEADING SECTIONS
%----------------------------------------------------------------------------------------
{\setstretch{1.1} 
\textsc{\LARGE {università degli studi di milano-bicocca}}\\[1cm] % Name of your university/college
}
\includegraphics[width = .25\textwidth]{logo_unimib.png}\\[1cm] % Include a department/university logo - this will require the gaphicx package
\textsc{ \large {Scuola di Scienze Matematiche Fisiche e Naturali}}\\[0.25cm]% Major heading such as course name
\textsc{\large {Corso di laurea Triennale in Fisica}}\\[0.75cm] % Minor heading such as course title

%----------------------------------------------------------------------------------------
%  TITLE SECTION
%----------------------------------------------------------------------------------------

%\HRule \\[0.4cm]
\vspace{1.5cm}
{\setstretch{1.5} 
{\large\bfseries PROBING OPTICAL AND RADIO-LOUD

 AGN FRACTIONS :

 A COMPARATIVE ANALYSIS BETWEEN BCGs 
 
AND NON-BCGs SAMPLES at z $<$ 0.1 }\\[0.5cm] % Title of your document
}
\vspace{3cm}
%\HRule \\[1.5cm]
 
%----------------------------------------------------------------------------------------
%  AUTHOR SECTION
%----------------------------------------------------------------------------------------
\begin{table}[htb!]
\centering
\begin{tabularx}{\textwidth}{X X}
\emph{Candidate:} & \emph{Supervisor:} \\
\textsc{Andrea Maccarinelli} & Prof. \textsc{Sebastiano Cantalupo}  \\
& \\
& \emph{Co-supervisors:} \\
& Dott. \textsc{Andrea Travascio} \\
\end{tabularx}
\end{table}

%\vspace{1cm}
%----------------------------------------------------------------------------------------
%  DATE SECTION
%----------------------------------------------------------------------------------------
%\setstretch{1.2} 
{
{\large \textsc{anno accademico 2022/2023}}\\[2cm] % Date, change the \today to a set date if you want to be precise
}

\vfill % Fill the rest of the page with whitespace

\end{document}
 



\begin{document}

%
\documentclass[a4]{article}
\usepackage[utf8]{inputenc}
\usepackage{mathtools}
\usepackage{listings}
\usepackage{verbatim}
\usepackage{graphicx}
\usepackage{setspace}
\usepackage{xcolor}
\usepackage{wrapfig}
%\usepackage[demo]{graphicx}
%\usepackage{caption}
\usepackage{subcaption}
\usepackage{caption}
\usepackage[utf8]{inputenc}
\usepackage{mathtools}
\usepackage{listings}
\usepackage{verbatim}
\usepackage{graphicx}
\graphicspath{ {./images/} }
\usepackage{float}
\usepackage{mwe}
\usepackage{tabularx}
\setlength{\headheight}{14.49998pt}
\addtolength{\topmargin}{-2.49998pt}

\newcommand{\HRule}{\rule{\linewidth}{0.5mm}} % Defines a new command for the horizontal lines, change thickness here

%\renewcommand{\baselinestretch}{1.05} 

\thispagestyle{empty}
\begin{document}



\center % Center everything on the page
 
%----------------------------------------------------------------------------------------
%  HEADING SECTIONS
%----------------------------------------------------------------------------------------
{\setstretch{1.1} 
\textsc{\LARGE {università degli studi di milano-bicocca}}\\[1cm] % Name of your university/college
}
\includegraphics[width = .25\textwidth]{logo_unimib.png}\\[1cm] % Include a department/university logo - this will require the gaphicx package
\textsc{ \large {Scuola di Scienze Matematiche Fisiche e Naturali}}\\[0.25cm]% Major heading such as course name
\textsc{\large {Corso di laurea Triennale in Fisica}}\\[0.75cm] % Minor heading such as course title

%----------------------------------------------------------------------------------------
%  TITLE SECTION
%----------------------------------------------------------------------------------------

%\HRule \\[0.4cm]
\vspace{1.5cm}
{\setstretch{1.5} 
{\large\bfseries PROBING OPTICAL AND RADIO-LOUD

 AGN FRACTIONS :

 A COMPARATIVE ANALYSIS BETWEEN BCGs 
 
AND NON-BCGs SAMPLES at z $<$ 0.1 }\\[0.5cm] % Title of your document
}
\vspace{3cm}
%\HRule \\[1.5cm]
 
%----------------------------------------------------------------------------------------
%  AUTHOR SECTION
%----------------------------------------------------------------------------------------
\begin{table}[htb!]
\centering
\begin{tabularx}{\textwidth}{X X}
\emph{Candidate:} & \emph{Supervisor:} \\
\textsc{Andrea Maccarinelli} & Prof. \textsc{Sebastiano Cantalupo}  \\
& \\
& \emph{Co-supervisors:} \\
& Dott. \textsc{Andrea Travascio} \\
\end{tabularx}
\end{table}

%\vspace{1cm}
%----------------------------------------------------------------------------------------
%  DATE SECTION
%----------------------------------------------------------------------------------------
%\setstretch{1.2} 
{
{\large \textsc{anno accademico 2022/2023}}\\[2cm] % Date, change the \today to a set date if you want to be precise
}

\vfill % Fill the rest of the page with whitespace

\end{document}
 





%\pagestyle{headings}

%\maketitle

%\pagestyle{empty}

%\tableofcontents


\pagestyle{fancyplain}
\addtolength{\headwidth}{\marginparsep}
\addtolength{\headwidth}{\marginparwidth}
% remember chapter title
\renewcommand{\chaptermark}[1]{\markboth{{\chaptername\ \thechapter}\ \--- #1}{}} %{\markboth{#1}{#1}}
% section number and title
\renewcommand{\sectionmark}[1]{\markright{\thesection\ \ #1}}
\lhead[\fancyplain{}{\thepage}]{\fancyplain{}{\emph\rightmark}}
\rhead[\fancyplain{}{\emph\leftmark}]{\fancyplain{}{\thepage}}
\cfoot{}


\documentclass[a4]{article}
\usepackage[utf8]{inputenc}
\usepackage{mathtools}
\usepackage{listings}
\usepackage{verbatim}
\usepackage{graphicx}
\usepackage{setspace}
\usepackage{xcolor}
\usepackage{wrapfig}
%\usepackage[demo]{graphicx}
%\usepackage{caption}
\usepackage{subcaption}
\usepackage{caption}
\usepackage[utf8]{inputenc}
\usepackage{mathtools}
\usepackage{listings}
\usepackage{verbatim}
\usepackage{graphicx}
\graphicspath{ {./images/} }
\usepackage{float}
\usepackage{mwe}
\usepackage{tabularx}
\setlength{\headheight}{14.49998pt}
\addtolength{\topmargin}{-2.49998pt}

\newcommand{\HRule}{\rule{\linewidth}{0.5mm}} % Defines a new command for the horizontal lines, change thickness here

%\renewcommand{\baselinestretch}{1.05} 

\thispagestyle{empty}
\begin{document}



\center % Center everything on the page
 
%----------------------------------------------------------------------------------------
%  HEADING SECTIONS
%----------------------------------------------------------------------------------------
{\setstretch{1.1} 
\textsc{\LARGE {università degli studi di milano-bicocca}}\\[1cm] % Name of your university/college
}
\includegraphics[width = .25\textwidth]{logo_unimib.png}\\[1cm] % Include a department/university logo - this will require the gaphicx package
\textsc{ \large {Scuola di Scienze Matematiche Fisiche e Naturali}}\\[0.25cm]% Major heading such as course name
\textsc{\large {Corso di laurea Triennale in Fisica}}\\[0.75cm] % Minor heading such as course title

%----------------------------------------------------------------------------------------
%  TITLE SECTION
%----------------------------------------------------------------------------------------

%\HRule \\[0.4cm]
\vspace{1.5cm}
{\setstretch{1.5} 
{\large\bfseries PROBING OPTICAL AND RADIO-LOUD

 AGN FRACTIONS :

 A COMPARATIVE ANALYSIS BETWEEN BCGs 
 
AND NON-BCGs SAMPLES at z $<$ 0.1 }\\[0.5cm] % Title of your document
}
\vspace{3cm}
%\HRule \\[1.5cm]
 
%----------------------------------------------------------------------------------------
%  AUTHOR SECTION
%----------------------------------------------------------------------------------------
\begin{table}[htb!]
\centering
\begin{tabularx}{\textwidth}{X X}
\emph{Candidate:} & \emph{Supervisor:} \\
\textsc{Andrea Maccarinelli} & Prof. \textsc{Sebastiano Cantalupo}  \\
& \\
& \emph{Co-supervisors:} \\
& Dott. \textsc{Andrea Travascio} \\
\end{tabularx}
\end{table}

%\vspace{1cm}
%----------------------------------------------------------------------------------------
%  DATE SECTION
%----------------------------------------------------------------------------------------
%\setstretch{1.2} 
{
{\large \textsc{anno accademico 2022/2023}}\\[2cm] % Date, change the \today to a set date if you want to be precise
}

\vfill % Fill the rest of the page with whitespace

\end{document}
 


\include{Abstract}
\tableofcontents
%\include{Abstract}
\include{introduzione}  
\chapter{Introduction}

\section{The Active Galactic Nuclei Feedback}

\section{The Brightest Cluster Galaxies }
One of the main branches of cosmic structure formation is covered by the Hierarchical model in which galaxies grow in both stellar and mass magnitude, by accreting the surrounding matter.

One of the most extreme examples in this context involves the study of Brightest Cluster Galaxies (BCGs), a unique class of galaxies typically elliptical, often situated at the center and typically standing out as the most luminous and massive objects within the entire cluster.( e.g. \cite{2015MNRAS.448....2W} ).

Considering the environment, also their evolution is proven via observational studies e.g. \cite{2020MNRAS.498.2719T}, to be slightly different to normal galactic ones, resulting in the current main model assumes that BCGs mass assembly is dominated by dry mergers  \cite{2007MNRAS.375....2D, 2019ApJ...881..150C}.


Thanks to their exclusive environment in which they live, consisting in a BCGs often host a Super Massive Black Hole ( SMBH ) (\cite{2006ApJ...652..216R}) resulting in Active Galactic Nuclei ( AGN ) activity such as Radio Jets that feedback into the Intra-Cluster Medium (ICM)\cite{2007ARA&A..45..117M}.

As seen in previous studies ( e.g. \cite{2007MNRAS.379..867V} ) BCGs more likely host Radio Loud Emission than in other type of galaxies, as a result of the influence of the environment other than its nuclei.


% Aggiungi parte sulle merging activity
% aggiungi parte su cosa si vede a bassi z e quali sono le teorie in questo senso che tengono banco
% Aggiungi dei riferimenti alle domande che restano aperte nel contesto.



%Thanks to the unique environment in which BCGs are located, made by over-dense 

\section{The Aim of this thesis}

    
\chapter{Methods}

\section{Data Description}
A description of the Data Samples Utilized, and its peculiarities, starting from The SDSS DR7 description, and following the work provided by the MPA-JHU team.
\section{Data Analysis}
How we analyzed the data, where do we defined the fractions calculated ?
\subsection{Optical analysis}
Describe each population of the BPT diagram and its peculiarities, before starting the description of how we collected and processed the data to finally create BPT diagrams .
Finally there's the need to explain how we interpreted the diagrams to finally calculate the fractions ( in this case you need to explain which algorithm has been chosen and why ) 


\subsection{Radio Analysis}
I certainly need to focus on how i have chosen the range of identification, to recognize which element of sdss was effectively recognized as a Radioloud.

Following to this it is necessary to explain how we selected the elements in which define the fractions, by selecting elements of SDSS nearby the location in which best et al finds mostly its radio-emitting elements!    
\chapter{Results}
At the conclusion of the analysis, a comparison between the results obtained in the two samples was conducted, and the outcomes are presented in the following tables.

In Table \ref{tab:Optical}, detailing the results of the optical analysis, it is observed that BCGs are more likely to host an AGN compared to ordinary galaxies. These findings align with those presented in the work of Vitale et al. \cite{2012A&A...546A..17V}, where the analysis was based on a statistically larger sample than the one employed in this study.

\vspace{1cm}
\begin{table}[htb]
  \centering
  \begin{tabular}{||c|ccc||}
    \hline\hline
    \multicolumn{1}{||c}{Diagram} & Spectral Type & SDSS DR7 subsample & Crossmatch C4-BCG \\
    \hline
    [NII]  & AGNs &  $52896 \pm 84$ ($21.1 \pm 0.03$)$\%$ &$161 \pm 5$ ($50.4 \pm 1.6$)$\%$ \\
           & Composites & $55875 \pm 110$ ($22.3 \pm 0.04$)$\%$ & $110 \pm 5$ ($34.5 \pm 1.8$)$\%$ \\
           & SFGs & $141753 \pm 80$ ($56.6 \pm 0.03$)$\%$ &$ 49 \pm 4$ ($15.4 \pm 1.4$)$\%$ \\ 
    \hline
    [SII]  & Seyferts & $13082 \pm 64$ ($6.3 \pm 0.03$)$\%$  &$8 \pm 2$ ($6.5 \pm 1.6$)$\%$\\
           & LINERs & $28396 \pm 80$ ($13.6 \pm 0.04$)$\%$ & $74 \pm 3$ ($53.3 \pm 2.4$)$\%$ \\
           & SFGs &$167826\pm 76 $ ($80.2 \pm 0.04$)$\%$ & $55 \pm 3$ ($40.2 \pm 2.2$)$\%$\\ 
    \hline\hline
  \end{tabular}
  \caption{Results of the optical analysis}
  \label{tab:Optical}
\end{table}



Simultaneously, the analysis of Radio Loud emissions indicates that BCGs are more likely to host Radio Loud Activity, with a fraction of $12\%$. This value is 20 times higher than the fraction found for the non-BCG subsample of galaxies within the selected regions, as discussed earlier, corresponding to $0.6\%$.
    
\chapter*{Summary and Conclusions}
\addcontentsline{toc}{chapter}{Summary and Conclusions}{}  % mette le conclusioni nell'indice

The evolutionary processes of BCGs are still not fully understood, and there are no specific
studies comparing the frequency of different types of AGN in BCGs with respect to other types of
galaxies (e.g., \cite{2019CoBAO..66..153F}).

The central inquiry driving this thesis project is: \textit{Does the unique evolutionary path of BCGs, shaped by their distinctive environmental conditions, contribute to heightened SMBH accretion at their cores compared to other galaxy types in the local universe?}

To tackle this scientific inquiry, this study conducts a comparative analysis on samples of Brightest Cluster Galaxies (BCGs) and non-BCGs. The samples are derived through the cross-matching of three distinct celestial catalogs, encompassing a comprehensive galaxy sample\cite{2009ApJS..182..543A}, a specific BCG sample \cite{2009yCat..73790867V}, and a radio survey identifying objects with pronounced radio emission \cite{2005MNRAS.362....9B}.

The analysis we carried out produced the following results:

\begin{enumerate}
\item BCGs show greater AGN activity than ordinary galaxies.
\item The unique environmental conditions within the cluster play a crucial role in shaping the distinctive form of AGN feedback. The preference for shock-induced gas interactions through radio jets is evident, leading to a higher percentage of Low-Ionization Nuclear Emission-line Region (LINER) phenomena compared to Seyfert galaxies.
\item In support of the preceding observation, the BCG sample exhibits a radio loudness fraction that is 20 times higher, providing further confirmation.
\end{enumerate}

Future enhancements to the methodology outlined in this study could involve incorporating a galaxy selection based also on mass magnitude. 

\bibliographystyle{plain}       
\bibliography{articoli}{}


%\include{bibliografia}         

% INIZIO COMMENTO
\begin{comment} 

% quello che sta qui non viene compilato

%\listoffigures

%\listoftables

\end{comment}
% FINE COMMENTO


\end{document}

