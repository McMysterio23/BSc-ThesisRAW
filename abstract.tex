\chapter*{Abstract}
This Thesis work focuses in inspecting how the different environment affects the way mass accretes onto a SMBH, and with that the correlated differences in the feedback.
The Aim of this work is to study samples of Brightest Cluster Galaxies ( BCGs ) and non-BCGs to inspect possible differences.

Such Analysis has been possible by cross-matching the Galaxy Sample created by the MPA-JHU team in the context of  "The Sloan Digital Sky Survey" ( SDSS ) with the C4 Cluster Catalogue, to identify the BCGs.  % AGGIUNGERE LE CITAZIONI \cite{miller_2005} e quella a Kauffman
To enlighten differences in how the environment affects the SMBH feedback, this work presents a comparative analysis made with AGN fractions defined in properly identified space regions, both in optical and in radio context.
To carry out a proper optical analysis, the study adopts the direction of a photo-spectrometry investigation, primarily due to the abundance of spectrum-derived data available in the SDSS-derived galaxy sample.
The following part of the analysis, focusing on radio-emitting objects, was conducted through a cross-match, this time involving a sample of radio emitter galaxies created by Best et al.

As a result, these analyses show that the fraction of Optical AGN is way greater in BCGs samples rather than noBCGs ones.
A further analysis of the BPT SII diagram also revealed that BCGs are cold objects whose AGN interacts more likely by shocking the gas rather than in a hot radiative way.
On the radio emission front, the analysis of the obtained data samples also discloses a higher percentage of RadioLoud emitting objects in the BCG sample compared to the other one.

In conclusion, this work demonstrates that, based on the collected data, there is no doubt that the warmer gas found in a galaxy cluster plays a crucial role in influencing the accretion of mass into the supermassive black hole (SMBH) of the Brightest Cluster Galaxy (BCG). Furthermore, for a more advanced study, it may be pertinent to explore the radio domain further, as suggested by the LINER classification that many BCGs have received.