ø\chapter*{Abstract}
\begin{comment}
Brightest cluster galaxies (BCGs) are the most massive and luminous galaxies located near the center of relaxed, virialized, and undisturbed galaxy clusters in the local Universe (\cite{1976ApJ...205..688S}; \cite{2010MNRAS.404.1231V}). 

According with several observational studies (\cite{2020MNRAS.498.2719T}) these objects experience special formation differing from galaxy evolution in general. Current theoretical models (e.g., \cite{2007MNRAS.375....2D}, \cite{2019ApJ...881..150C}) predict that the dry mergers are the dominant mechanisms responsible for their mass assembly.

Often, these objects are observed to host a super-massive black hole (SMBH) in their center (\cite{2006ApJ...652..216R}).  
The process of matter accretion into these SMBHs may release a large amount of energy, resulting in Active Galactic Nuclei (AGN).
There are two primary modes in which SMBH accretion can occur: the so-called 'radio mode' and the 'quasar mode'
The quasar mode consists in a high accretion rate of the SMBH via an optically-thick and geometrically-thin disk, and most of the energy is released in form of radiation. In a radio mode scenario, the SMBH accretion of hotter gas happens with a low rate in a optically-thin and geometrically-thick disk configuration, releasing energy in form of relativistic particles. The latter is typically observed in BCG. 

%The fraction of AGN activity in BCGs has been explored in previous papers (\cite{2016MNRAS.460.3669Y}, \cite{2014MNRAS.440..762O}), showing that the radio-loud fractions increase together with the source magnitude, while on the optical side, BCGs are more likely to host AGN activity rather than Star Forming Character.

Therefore, in this thesis i present an analysis of Optical and Radio Loud AGN fractions on 2 galaxy samples derived from the combination of the galaxy sample created by the MPA-JHU team in the context of  "The Sloan Digital Sky Survey" ( SDSS ),  the C4 BCGs catalogue by \cite{2005AJ....130..968M}, and the Radio Emitters catalogue developed by \cite{2005MNRAS.362....9B}.

As the Optical activity is classified through the BPT diagram diagnostic, the sample without the BCGs has been obtained by overlapping space regions covered by both the SDSS survey and the Radio Emitters one, to achieve a proper confrontation between the results in the different samples.

These analyses reveal that based on a optical point of view BCGs show a greater AGN activity $\sim 60 \%$ when compared to the $\sim 16\%$ obtained on the other sample similar to the ones obtained by \cite{2012A&A...546A..17V}.
While At the same time Radio Loud inspections show that the fractions of AGN objects is similar between the BCG sample ($\sim 62 \%$) and the other ( $\sim 68\%$)

In conclusion, these results strongly support the idea that BCGs are more likely to exhibit radio loudness, as observed in previous studies such \cite{2016MNRAS.460.3669Y, 2014MNRAS.440..762O}. Therefore, any future investigations in this context should prioritize a radio analysis over an optical one.
\end{comment}


Brightest cluster galaxies (BCGs) are the most massive and luminous galaxies located near the center of relaxed, virialized, and undisturbed galaxy clusters in the local Universe (\cite{1976ApJ...205..688S, 2010MNRAS.404.1231V}.
According to several observational studies (\cite{2017MNRAS.467.4101G}, \cite{2020MNRAS.498.2719T}), these objects experience a special formation process differing from general galaxy evolution. Current theoretical models (e.g., \cite{2007MNRAS.375....2D, 2019ApJ...881..150C}) predict that dry mergers are the dominant mechanisms responsible for their mass assembly at $z<1$.
These objects are often observed to host a supermassive black hole (SMBH) in their center (\cite{2006ApJ...652..216R}). The process of matter accretion into these SMBHs may release a large amount of energy, resulting in Active Galactic Nuclei (AGN). There are two primary modes in which SMBH accretion can occur: the so-called 'quasar mode' and the 'radio mode'. The quasar mode involves a high accretion rate of the SMBH via an optically-thick and geometrically-thin disk, with most of the energy being released in the form of radiation. In a radio mode scenario, the SMBH accretion of hotter gas occurs at a low rate in an optically-thin and geometrically-thick disk configuration, releasing energy in the form of relativistic particles, i.e. radio jets. The latter is typically observed in BCGs.

The evolutionary processes of BCGs are still not fully understood, and there are no specific studies comparing the frequency of different types of AGN in BCGs with respect to other types of galaxies (e.g., \cite{2019CoBAO..66..153F}).

The main scientific question driving the project of my thesis is to understand whether the different evolution of BCGs, along with their "special" environment, affects the accretion of SMBHs in their centers compared to other types of galaxies in the local universe, at $z<0.1$. To address this question, I analyzed a sample of BCGs derived from the combination of the Sloan Digital Sky Survey Data Release 7 (SDSS DR7; described in \cite{2009ApJS..182..543A}) and the C4 BCGs catalogue produced by \cite{2005AJ....130..968M}. For this BCG sample at z=0.02-0.1, I use the fluxes of optical lines ($H\alpha$, [OIII], $H\beta$, [NII], and [SII] doublets) obtained by the MPA-JHU team using methods described in \cite{1994ApJ...422..158O} to obtain a census of optical AGN through the NII and SII BPT diagnostic diagrams. Additionally, by cross-matching the previous catalogs with a collection derived from NVSS and FIRST radio surveys, as presented in \cite{2005MNRAS.362....9B}, I identify the BCGs exhibiting radio loud emission, implying the presence of radio jets.

Following this classification, I finally estimate the fraction of BCGs classified as Optical and Radio Loud AGN. Subsequently, I derive the same fractions for the non-BCG selected galaxies according to the cross-match with the C4 catalogue  (\cite{2005AJ....130..968M}.

The analyses reveal that BCGs exhibit a higher fraction of optical AGN $\sim50\%$ compared to the non-BCG sample, which shows a percentage of $\sim 21\%$, consistent with results found by \cite{2012A&A...546A..17V}.
Simultaneously, the analysis of Radio Loud emissions indicates that BCGs are more inclined to host Radio Loud Activity, with a fraction of  $\sim 12 \%$. This fraction is found to be 20 times higher than the $\sim 0.6\%$ observed in the non-BCG sample of selected galaxies.

In conclusion, these results strongly support the idea that BCGs are more likely to exhibit typical ionization due to AGN activity and radio emission. This implies that their privileged position supports frequent accretion of SMBHs in both accretion modes. Previous studies, such as \cite{2016MNRAS.460.3669Y, 2014MNRAS.440..762O, 2007MNRAS.379..867V}, have already shown a prevalence of radio AGN among the BCG population in the local Universe.  
On the other hand, the larger fraction of BCGs selected as optical AGN compared to normal galaxies is intriguing. Future studies will aim to test the results obtained with this sample and understand if this higher fraction is specifically driven by differences in properties between BCGs and non-BCGs, such as mass, star formation rate, metallicity, and kinematics.



%RESTA DA AGGIORNARE QUESTA VERSIONE A QUELLA PRESENTE NEL FILE DEFINITIVO,
%ED AGGIUNGERE LE CITAZIONI A BALDWIN ET AL PER I BPT DIAGRAMS, ED A Becker, White & Helfand 1995 PER IL CATALOGO DELLE RADIO EMITTERS ed a Condon et al. 1998 per il NVSS sempre per il catalogo delle radio emitters.
