\chapter*{Ringraziamenti}
Desidero ringraziare il mio relatore, Sebastiano Cantalupo, per avermi dato la possibilità di realizzare questa tesi in un ambito che da sempre mi appassiona, con un progetto cucito su misura sui miei interessi. Mi sento di ringraziare ancor di più il mio correlatore, Andrea Travascio, una persona brillante che mi ha saputo trasmettere la passione che lo guida ogni giorno nella sua vita quotidiana di ricercatore e da cui ho imparato moltissimo ed in diversi contesti, è stato davvero un piacere lavorare con te !!

Infine desidero ringraziare tutti coloro che mi sono stati vicini in questi anni, a partire dalla mia famiglia che non ha mai smesso di mettermi alla prova ed a mio fratello che ha sempre rispettato le mie scelte. 
Un grazie a Greg ed alla mia squadra di Hockey, protagonisti di una parte importantissima della mia vita, che non smetterò mai di ricordare con immenso piacere.
E poi ci siete VOI, \textbf{TUTTI VOI}, amici di Bergamo, Milano e dintorni, del gruppo "Tautologia" e non : dal primo all’ultimo grazie per tutti gli splendidi momenti vissuti insieme, vorrei sicuramente avervi conosciuto prima. Anche se non riuscirò a nominare tutti, ci tengo a dare una menzione d'onore a Samuele grande estimatore della Frenchcore olandese, che definire amico è assolutamente riduttivo; a Giorgio per avermi aiutato a superare un brutto momento quella sera di Novembre e per tutte le volte che abbiamo parlato liberamente; a Margherita con la quale ci siamo fatti forza a vicenda per superare l'ultimo ostacolo e la cui energia è veramente contagiosa; a Maso una delle menti più brillanti che abbia conosciuto, da cui ho imparato tanto ed in diversi contesti, a Davide che è un piacere avere nella mia vita e con cui non c'è risata collettiva che si possa definire breve; a Monny, un bomber con cui condivido tantissimo e che ho avuto il piacere di conoscere solo in università a Milano nonostante sia sempre stato davanti a casa mia; ad Alessandro, un amico vero, il miglior compagno di viaggio per andare a scoprire il mondo; a Rocco grande estimatore del dibattito politico con cui è sempre un piacere condividere punti di vista; a Simone che mi ha trasmesso il piacere dei piccoli momenti, mai scontato di questi tempi; ed a Lorenzo da cui è partito tutto all'inizio di questi 3 anni e che sempre rimarrà un amico ed il primo \textbf{collega} di laboratorio della mia carriera accademica.

Arrivo dunque al più grande ringraziamento che ci tengo a fare: 

\begin{center}
\textbf{Grazie Nonno Mario !!}
\end{center}

Anche se non sei qui con me fisicamente per questo mio primo grande traguardo, mi sento di dedicarti questa conquista, perché mai ho smesso di pensare agli insegnamenti che mi hai impartito.
Sei la stella più luminosa che mi ha sempre aiutato a ritrovare il sentiero nei momenti di disorientamento che ci sono stati, ci sono e che sicuramente la vita mi presenterà in futuro.
Nei pochi anni che ti ho vissuto, mi hai trasmesso tantissimo, soprattutto il culto dell'ordine e della disciplina, elemento fondamentale per superare gli ostacoli della vita di tutti i giorni e per raggiungere obbiettivi sempre più ambiziosi. Sento che l'eterna fonte della mia ambizione sia frutto di quanto mi hai saputo trasmettere sulla capacità di saper fare dei sacrifici per un obiettivo, 


\vspace{2cm}

\textbf{IL MIO OBIETTIVO.}

\begin{flushright}
\textit{Per Aspera Ad Astra}
\end{flushright}