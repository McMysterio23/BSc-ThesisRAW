\chapter*{Summary and Conclusions}
\addcontentsline{toc}{chapter}{Summary and Conclusions}{}  % mette le conclusioni nell'indice

The evolutionary processes of BCGs are still not fully understood, and there are just few
studies comparing the frequency of different types of AGN in BCGs based on BPT diagram analysis (e.g., \cite{2019CoBAO..66..153F}).

The central inquiry driving this thesis project is: \textit{Does the unique evolutionary path of BCGs, shaped by their distinctive environmental conditions, contribute to heightened SMBH accretion at their cores compared to other galaxy types in the local universe?}

To tackle this scientific inquiry, this study conducts a comparative analysis on samples of Brightest Cluster Galaxies (BCGs) and non-BCGs. The samples are derived through the cross-matching of three distinct catalogs, encompassing a comprehensive sample of galaxies \cite{2009ApJS..182..543A}, catalogs of selected BCGs  \cite{2009yCat..73790867V}, and radio-loud sources \cite{2005MNRAS.362....9B} associated to these galaxies, in a specific overlapped area shown in \autoref{8}.

Using the spectroscopic properties of the optical lines derived from MPA-JHU team in the context of SDSS DR7 \cite{mpa-sdss-dr7, 2009ApJS..182..543A}, I estimated the fraction of Optical AGN, and radio emission from AGN activity, in samples of BCGs and non-BCGs and I found following results.
Under the assumption that all objects in the BPT-NII and -SII diagrams are AGN, I found that $\sim75-90\%$ of the BCG host an AGN, over a sample refined over SNR greater than 3.
Compared with non-BCG in the same redshift range, with the same requirement in SNR, i detected a $\sim 22-55\%$ resulting in a value $60\%$ smaller than the previous. The higher AGN activity observed in the BCG galaxies, situated at the center of clusters by definition, as derived from the BPT analyses, contrasts with some studies that have found a typically higher AGN fraction in the field compared to clusters (i.e. \cite{2017MNRAS.472..409L}).  On the other hand, it agrees with other studies, also based on BPT diagram analyses ( i.e. \cite{2012A&A...538A..15H}).  The fact that BCGs exhibit more common AGN activity can be attributed to the large amount of hot gas, typically detected as X-rays in galaxy clusters' ICM , which cools and falls into the BCG, fueling both galaxy growth and the supermassive black hole in previous stages \cite{2012A&A...538A..15H}. 

The fraction of BCGs that exhibit Radio Loud emission is 20x higher than the value resulted for non-BCGs galaxies.
This evidence may suggest that the unique environmental conditions within the cluster play a crucial role in shaping the accretion mode of SMBH at the center of the BCG, as well as the subsequent AGN feedback, as widely shown in literature (i.e.\cite{2019ApJ...883..193N, 2021A&A...649A..23C}).
The same conclusion can be achieved by looking at the results showing an higher percentage of LINERs in the BPT-SII diagram. Indeed, LINERs are objects in which the low-ionization nuclear emission lines such as [NII] and [SII] are excited. Some authors ( i.e. \cite{Fabbiano_2022}) suggest that these high excitation are a consequence of shock-induced gas interactions through radio jets.

The scope of this thesis could be extended by delving into the intrinsic properties of BCGs that might be intricately associated with the presence of an AGN or radio loud emissions. This entails a comprehensive exploration of data provided by the MPA-JHU to gain insights into the intrinsic characteristics of BCG populations more prone to hosting AGN activities, including factors such as mass, star formation rate (SFR), and redshift. Such analysis would not only deepen our understanding but also contribute to a broader comparative perspective, aligning with the methodology employed in this study. To accomplish this, a logical progression would involve a more in-depth examination of the two distinct samples already established – one comprising BCGs and the other non-BCGs.

\begin{comment}
Lascio a te scrivere per bene la parte di prospettive future su questa linea di ricerca. Io qui ti faccio un discorsetto in italiano, tu riportalo per bene a modo tuo.
Ora abbiamo visto se le BCG hanno una significativa differenza in frazione di AGN rispetto alle non BCG giusto? E al momento quello che possiamo fare è semplicemente ipotizzare che queste differenze siano guidate dai differenti processi evolutivi delle BCG, a loro volta come conseguenza del fatto che queste si trovano alcentro di overdensity, in cui gas accretion dal cosmic web e merger activity sono all’ordine del giorno. 

Tuttavia sarebbe interessante capire se ci sono proprietà specifiche di queste BCG, che sono correlate con la presenza di un AGN o di un’emissione radio.  Per fare ciò abbiamo altre proprietà derivate dal team XX (gli stessi che ti danno i redshifts e le emissioni in riga), che possono essere utilizzate per avere una view del tipo di proprietà che ha la popolazione più incline ad avere un attività di AGN (Massa, SFR, Z, sigma della riga). Quindi in futuro sarà importante investigare eventuali correlazioni, come già è stato fatto in altri lavori (cercateli), però in questo caso andiamo a studiare le correlazioni in funzione della selezione che abbiamo di BCG e non-BCG. (ps- questi subsamples sono in sè già una selezione in Massa e Environment, perché ovviamente le BCG saranno le galassie più massicce in regioni dense).
Detto ciò vedi quello che puoi scrivere.
Buona conclusione.
\end{comment}



\begin{comment}
\begin{enumerate}
\item BCGs show greater AGN activity than ordinary galaxies.
\item The unique environmental conditions within the cluster play a crucial role in shaping the distinctive form of AGN feedback. The preference for shock-induced gas interactions through radio jets is evident, leading to a higher percentage of Low-Ionization Nuclear Emission-line Region (LINER) phenomena compared to Seyfert galaxies.
\item In support of the preceding observation, the BCG sample exhibits a radio loudness fraction that is 20 times higher, providing further confirmation.
\end{enumerate}

Future enhancements to the methodology outlined in this study could involve incorporating a galaxy selection based also on mass magnitude.
\end{comment}