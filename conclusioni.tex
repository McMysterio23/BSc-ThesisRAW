\chapter*{Summary and Conclusions}
\addcontentsline{toc}{chapter}{Summary and Conclusions}{}  % mette le conclusioni nell'indice

The evolutionary processes of BCGs are still not fully understood, and there are no specific
studies comparing the frequency of different types of AGN in BCGs with respect to other types of
galaxies (e.g., \cite{2019CoBAO..66..153F}).

The central inquiry driving this thesis project is: \textit{Does the unique evolutionary path of BCGs, shaped by their distinctive environmental conditions, contribute to heightened SMBH accretion at their cores compared to other galaxy types in the local universe?}

To tackle this scientific inquiry, this study conducts a comparative analysis on samples of Brightest Cluster Galaxies (BCGs) and non-BCGs. The samples are derived through the cross-matching of three distinct celestial catalogs, encompassing a comprehensive galaxy sample\cite{2009ApJS..182..543A}, a specific BCG sample \cite{2009yCat..73790867V}, and a radio survey identifying objects with pronounced radio emission \cite{2005MNRAS.362....9B}.

The analysis we carried out produced the following results:

\begin{enumerate}
\item BCGs show greater AGN activity than ordinary galaxies.
\item The unique environmental conditions within the cluster play a crucial role in shaping the distinctive form of AGN feedback. The preference for shock-induced gas interactions through radio jets is evident, leading to a higher percentage of Low-Ionization Nuclear Emission-line Region (LINER) phenomena compared to Seyfert galaxies.
\item In support of the preceding observation, the BCG sample exhibits a radio loudness fraction that is 20 times higher, providing further confirmation.
\end{enumerate}

Future enhancements to the methodology outlined in this study could involve incorporating a galaxy selection based also on mass magnitude.